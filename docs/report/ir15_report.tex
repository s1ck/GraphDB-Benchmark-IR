\documentclass[11pt, a4paper, oneside, twocolumn]{article} % article scrartcl
\usepackage[utf8x]{inputenc}		% Umlaute können genutzt werden ohne \" davor zu stellen
\usepackage{ucs}					% contains support for using UTF-8 as input encoding
\usepackage{amsmath}				% besser Mathe-Ausgabe
\usepackage{amsfonts}				% Mathe-Schrift
\usepackage{amssymb}				% Mathe-Symbole
\usepackage[ngerman]{babel}			% für deutsche Sprache. Übersetzungen (z.Bsp. Zusammenfassung <-- Abstract) und Silbentrennung 
\usepackage{fancyhdr}				% für Kopf- und Fußzeilen
\usepackage{graphicx}				% für Bilder
\usepackage{graphics}				% für Bilder
\usepackage{url}					% für anklickbare URLs
\usepackage{parskip}				% kein Einrücken sondern Leerzeilen zwischen Abschnitten
\usepackage{color}					% Farben
\usepackage{units}   				% Einheiten, \unit[23]{m}
\usepackage{hyperref}				% Referenzen

\author{Martin Junghanns \\  \url{martin.junghanns@studserv.uni-leipzig.de} \and 
		Sascha Ludwig \\ \url{s.ludwig@studserv.uni-leipzig.de} \and 
		Robert Schulze \\ \url{robert.schulze@studserv.uni-leipzig.de} }
\date{\today}
\title{Performanz Evaluation von Graphdatenbanksystemen versus SQL am Beispiel von Kookkurrenzgraphen }

\begin{document}

% bei Aufzählungen einen Strich an Stelle eines Punktes verwenden
\renewcommand{\labelitemi}{-}

% Titel mit Autoren
\maketitle

% Abstract
\begin{abstract}
	In dieser Arbeit wurden...    The research focus (i.e. statement of the problem(s)/research issue(s) addressed);
    The research methods used (experimental research, case studies, questionnaires, etc.);
    The results/findings of the research; and
    The main conclusions and recommendations

\end{abstract}

% der wahre Inhalt
\section{Neo4j}

\section{OrientDB}


\section{DEX}


\end{document}