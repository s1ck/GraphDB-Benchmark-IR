\documentclass[11pt, a4paper, oneside, twocolumn]{article} % article scrartcl
\usepackage[utf8x]{inputenc}		% Umlaute können genutzt werden ohne \" davor zu stellen
\usepackage{ucs}					% contains support for using UTF-8 as input encoding
\usepackage{amsmath}				% besser Mathe-Ausgabe
\usepackage{amsfonts}				% Mathe-Schrift
\usepackage{amssymb}				% Mathe-Symbole
\usepackage[ngerman]{babel}			% für deutsche Sprache. Übersetzungen (z.Bsp. Zusammenfassung <-- Abstract) und Silbentrennung 
\usepackage{fancyhdr}				% für Kopf- und Fußzeilen
\usepackage{graphicx}				% für Bilder
\usepackage{graphics}				% für Bilder
\usepackage{url}					% für anklickbare URLs
\usepackage{parskip}				% kein Einrücken sondern Leerzeilen zwischen Abschnitten
\usepackage{color}					% Farben
\usepackage{units}   				% Einheiten, \unit[23]{m}
\usepackage{hyperref}				% Referenzen

\author{Martin Junghanns \\  \url{martin.junghanns@studserv.uni-leipzig.de} \and 
		Sascha Ludwig \\ \url{s.ludwig@studserv.uni-leipzig.de} \and 
		Robert Schulze \\ \url{robert.schulze@studserv.uni-leipzig.de} }
\date{\today}
\title{Benchmark für Graphdatenbanken}

\begin{document}

% bei Aufzählungen einen Strich an Stelle eines Punktes verwenden
\renewcommand{\labelitemi}{-}

% Titel mit Autoren
\maketitle

% Abstract
\begin{abstract}
	Dieser Report stellt unsere Vergleiche von Graphdatenbanken dar. Hierbei wurden ...
\end{abstract}

% der wahre Inhalt
\section{Neo4j}
\subsection{Anbindung}


Abgesehen von mechanischen Phrasendreschmaschinen als Vorläufern und abgesehen von frühesten Versuchen, Texte durch Software zu generieren, beginnt die erste Phase natürlichsprachiger Generierung mit Programmen, die zur Textgenerierung schematisch auf Wissen zugreifen, das bereits in Textform abgelegt ist. So funktionierte ab 1963 BASEBALL, ein Interface zu den Baseballdaten der amerikanischen Baseballiga und SAD SAM, ein Interface zur Eingabe von Verwandtschaftsbeziehungen, das bereits auf Fragen antwortete. Nach mehreren anderen Arbeiten in dieser Richtung erschien 1966 ELIZA, programmiert von Joseph Weizenbaum. In der zweiten Phase ist das Wissen in Fakten und Regeln kodiert: LUNAR, 1972, ist das Interface zur Datenbank über die Mondprobensammlung der Apollo 11 Mission. PARRY, 1975, simuliert einen Paranoiden in Gespräch mit einem Psychiater. ROBOT, 1977, ist das erste kommerzielles Frage-Antwort-System. VIE-LANG, 1982, von Ernst Buchberger, ist ein Dialogsystem in deutscher Sprache, das Sätze aus einem semantischen Netz generiert[14]. HAM-ANS, 1983, von Wolfgang Hoeppner, ist ein Dialogsystem in deutscher Sprache, das beispielsweise einen Hotelmanager simuliert.

Abgesehen von mechanischen Phrasendreschmaschinen als Vorläufern und abgesehen von frühesten Versuchen, Texte durch Software zu generieren, beginnt die erste Phase natürlichsprachiger Generierung mit Programmen, die zur Textgenerierung schematisch auf Wissen zugreifen, das bereits in Textform abgelegt ist. So funktionierte ab 1963 BASEBALL, ein Interface zu den Baseballdaten der amerikanischen Baseballiga und SAD SAM, ein Interface zur Eingabe von Verwandtschaftsbeziehungen, das bereits auf Fragen antwortete. Nach mehreren anderen Arbeiten in dieser Richtung erschien 1966 ELIZA, programmiert von Joseph Weizenbaum. In der zweiten Phase ist das Wissen in Fakten und Regeln kodiert: LUNAR, 1972, ist das Interface zur Datenbank über die Mondprobensammlung der Apollo 11 Mission. PARRY, 1975, simuliert einen Paranoiden in Gespräch mit einem Psychiater. ROBOT, 1977, ist das erste kommerzielles Frage-Antwort-System. VIE-LANG, 1982, von Ernst Buchberger, ist ein Dialogsystem in deutscher Sprache, das Sätze aus einem semantischen Netz generiert[14]. HAM-ANS, 1983, von Wolfgang Hoeppner, ist ein Dialogsystem in deutscher Sprache, das beispielsweise einen Hotelmanager simuliert.

Abgesehen von mechanischen Phrasendreschmaschinen als Vorläufern und abgesehen von frühesten Versuchen, Texte durch Software zu generieren, beginnt die erste Phase natürlichsprachiger Generierung mit Programmen, die zur Textgenerierung schematisch auf Wissen zugreifen, das bereits in Textform abgelegt ist. So funktionierte ab 1963 BASEBALL, ein Interface zu den Baseballdaten der amerikanischen Baseballiga und SAD SAM, ein Interface zur Eingabe von Verwandtschaftsbeziehungen, das bereits auf Fragen antwortete. Nach mehreren anderen Arbeiten in dieser Richtung erschien 1966 ELIZA, programmiert von Joseph Weizenbaum. In der zweiten Phase ist das Wissen in Fakten und Regeln kodiert: LUNAR, 1972, ist das Interface zur Datenbank über die Mondprobensammlung der Apollo 11 Mission. PARRY, 1975, simuliert einen Paranoiden in Gespräch mit einem Psychiater. ROBOT, 1977, ist das erste kommerzielles Frage-Antwort-System. VIE-LANG, 1982, von Ernst Buchberger, ist ein Dialogsystem in deutscher Sprache, das Sätze aus einem semantischen Netz generiert[14]. HAM-ANS, 1983, von Wolfgang Hoeppner, ist ein Dialogsystem in deutscher Sprache, das beispielsweise einen Hotelmanager simuliert.

\end{document}